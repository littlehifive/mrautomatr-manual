% Options for packages loaded elsewhere
\PassOptionsToPackage{unicode}{hyperref}
\PassOptionsToPackage{hyphens}{url}
%
\documentclass[
]{book}
\usepackage{lmodern}
\usepackage{amsmath}
\usepackage{ifxetex,ifluatex}
\ifnum 0\ifxetex 1\fi\ifluatex 1\fi=0 % if pdftex
  \usepackage[T1]{fontenc}
  \usepackage[utf8]{inputenc}
  \usepackage{textcomp} % provide euro and other symbols
  \usepackage{amssymb}
\else % if luatex or xetex
  \usepackage{unicode-math}
  \defaultfontfeatures{Scale=MatchLowercase}
  \defaultfontfeatures[\rmfamily]{Ligatures=TeX,Scale=1}
\fi
% Use upquote if available, for straight quotes in verbatim environments
\IfFileExists{upquote.sty}{\usepackage{upquote}}{}
\IfFileExists{microtype.sty}{% use microtype if available
  \usepackage[]{microtype}
  \UseMicrotypeSet[protrusion]{basicmath} % disable protrusion for tt fonts
}{}
\makeatletter
\@ifundefined{KOMAClassName}{% if non-KOMA class
  \IfFileExists{parskip.sty}{%
    \usepackage{parskip}
  }{% else
    \setlength{\parindent}{0pt}
    \setlength{\parskip}{6pt plus 2pt minus 1pt}}
}{% if KOMA class
  \KOMAoptions{parskip=half}}
\makeatother
\usepackage{xcolor}
\IfFileExists{xurl.sty}{\usepackage{xurl}}{} % add URL line breaks if available
\IfFileExists{bookmark.sty}{\usepackage{bookmark}}{\usepackage{hyperref}}
\hypersetup{
  pdftitle={TIES Measurement Report Automation Project},
  pdfauthor={Michael Wu},
  hidelinks,
  pdfcreator={LaTeX via pandoc}}
\urlstyle{same} % disable monospaced font for URLs
\usepackage{longtable,booktabs}
\usepackage{calc} % for calculating minipage widths
% Correct order of tables after \paragraph or \subparagraph
\usepackage{etoolbox}
\makeatletter
\patchcmd\longtable{\par}{\if@noskipsec\mbox{}\fi\par}{}{}
\makeatother
% Allow footnotes in longtable head/foot
\IfFileExists{footnotehyper.sty}{\usepackage{footnotehyper}}{\usepackage{footnote}}
\makesavenoteenv{longtable}
\usepackage{graphicx}
\makeatletter
\def\maxwidth{\ifdim\Gin@nat@width>\linewidth\linewidth\else\Gin@nat@width\fi}
\def\maxheight{\ifdim\Gin@nat@height>\textheight\textheight\else\Gin@nat@height\fi}
\makeatother
% Scale images if necessary, so that they will not overflow the page
% margins by default, and it is still possible to overwrite the defaults
% using explicit options in \includegraphics[width, height, ...]{}
\setkeys{Gin}{width=\maxwidth,height=\maxheight,keepaspectratio}
% Set default figure placement to htbp
\makeatletter
\def\fps@figure{htbp}
\makeatother
\setlength{\emergencystretch}{3em} % prevent overfull lines
\providecommand{\tightlist}{%
  \setlength{\itemsep}{0pt}\setlength{\parskip}{0pt}}
\setcounter{secnumdepth}{5}
\usepackage{booktabs}
\usepackage{amsthm}
\makeatletter
\def\thm@space@setup{%
  \thm@preskip=8pt plus 2pt minus 4pt
  \thm@postskip=\thm@preskip
}
\makeatother
\ifluatex
  \usepackage{selnolig}  % disable illegal ligatures
\fi
\usepackage[]{natbib}
\bibliographystyle{apalike}

\title{TIES Measurement Report Automation Project}
\author{Michael Wu}
\date{2021-07-09}

\begin{document}
\maketitle

{
\setcounter{tocdepth}{1}
\tableofcontents
}
\hypertarget{project-overview}{%
\chapter{Project overview}\label{project-overview}}

This project is developed to generate lengthy but informative measurement reports from survey data and Mplus measurement model outputs for projects at \href{https://steinhardt.nyu.edu/ihdsc/global-ties}{NYU Global TIES for Children}.

Typically, a research institute like TIES has the obligation to generate detailed measurement reports to better inform the funders and the cooperating agencies about its most up-to-date work. However, even with a Word template, the process from analytical results to a publishable report is \underline{\textbf{unnecessarily inefficent and prone to mistakes}} even for the most careful research assistant.

Therefore, we develop an R package called \href{https://github.com/nyuglobalties/mrautomatr}{\texttt{mrautomatr}} to be used in conjunction with Rstudio to address this issue. Currently, the project only suits the need of NYU Global TIES, where we can impose \href{https://nyu.box.com/s/ate5l7wmw164u7xjg3g8x1vrfhwnt0ax}{naming rules} for files and variables and most people use Mplus for measurement modeling and STATA for other analyses. Future adaptations are needed as people move their analysis to R.

The project workflow is shown below: the users specify parameters in a Microsoft Excel sheet and move several files to a destined folder, run a command in R, and (\textbf{boom!}) there is a well-formatted measurement report in Microsoft Word (powered by the \href{https://davidgohel.github.io/flextable/}{\texttt{flextable}} \& \href{https://bookdown.org/yihui/rmarkdown/}{\texttt{rmarkdown}} R packages).

\begin{center}\includegraphics[width=0.66\linewidth]{images/mrautomatr_diagram} \end{center}

We chose Word over LaTex (which generates pdf files) and html (which generates web pages) simply to minimize the confusion around writing codes in R, which takes a long time to learn. \textbf{Our hope is that even if you are new to R, you'll still be able to use the package efficiently for your own report generation}. After generating the reproducible parts of the report, feel free to rename it and manually edit the sections that are text-heavy. Check out an example rendered report \href{https://nyu.box.com/s/wbwcqj3u2x9y6levrwgn9przp7ptr5hp}{here}.

Check out the \href{https://github.com/nyuglobalties/workshops}{TIES R workshop series} and Yihui Xie's \href{https://bookdown.org/yihui/rmarkdown/}{Rmarkdown book} if you'd like to learn some necessary tools to customize your reporting formats using R.

\hypertarget{set-up-the-package}{%
\chapter{Set up the package}\label{set-up-the-package}}

\hypertarget{install-the-necessary-softwares}{%
\section{Install the necessary softwares}\label{install-the-necessary-softwares}}

You need to set up R and Rstudio on your computer before everything. R is the programming language that powers this project, and Rstudio is the interface that allows you better interact with your R code. Please follow the steps below:

\begin{itemize}
\tightlist
\item
  Download R \href{https://cran.r-project.org/mirrors.html}{here} and install it before you install Rstudio.
\item
  Download Rstudio \href{https://rstudio.com/products/rstudio/download/\#download}{here} and install it.
\item
  Open Rstudio, and click the first icon from the left on the Rstudio toolbar, and select R Markdown. Rstudio will prompt you to install several packages, just follow the instructions and install them.
\end{itemize}

\begin{center}\includegraphics[width=0.66\linewidth]{images/Rstudio_toolbar} \end{center}

\hypertarget{download-and-install-the-mrautomatr-package}{%
\section{\texorpdfstring{Download and install the \texttt{mrautomatr} package}{Download and install the mrautomatr package}}\label{download-and-install-the-mrautomatr-package}}

\begin{itemize}
\tightlist
\item
  Run the following lines:
\end{itemize}

\begin{verbatim}
install.packages("usethis")
install.packages("devtools")
library(usethis)
library(devtools)
\end{verbatim}

\begin{itemize}
\item
  You will need to set up your GitHub Personal Auth Token because this package is still internal and private at this point. You may need to email Michael Wu (\href{mailto:zw1429@nyu.edu}{\nolinkurl{zw1429@nyu.edu}}) in order to gain access to the TIES github repository.
\item
  Essentially, You need to set up your personal access token in a file called \texttt{.Renviron}. After you install the \texttt{usethis} package, run \texttt{usethis::create\_github\_token()}.
\item
  It'll take you through the process of creating a token, and then GitHub will give you a string of characters. \textbf{Copy that text to somewhere safe for now.}
\item
  Then, back in R, run \texttt{usethis::edit\_r\_environ()} and add this to the file: \texttt{GITHUB\_PAT={[}the\ token\ text\ here\ without\ the\ brackets{]}}. For example, \texttt{GITHUB\_PAT=ghp\_...}.
\item
  Save that file and restart your R session for changes to take effect.
\end{itemize}

\begin{center}\includegraphics[width=0.66\linewidth]{images/restart_R} \end{center}

\begin{itemize}
\item
  To verify that you got it to work, when R starts again, run \texttt{Sys.getenv("GITHUB\_PAT")} and you should see your token, exactly how it was shown on GitHub. If you don't, more troubleshooting is required and please email Michael.
\item
  Run the following line:
\end{itemize}

\begin{verbatim}
devtools::install_github("nyuglobalties/mrautomatr")
library(mrautomatr)
\end{verbatim}

\begin{itemize}
\tightlist
\item
  Check out the functions by running \texttt{?function\_name}, e.g.:
\end{itemize}

\begin{verbatim}
?mrautomatr
\end{verbatim}

You should be able to see the documentation page of \texttt{mrautomatr} in the Help panel.

\begin{center}\includegraphics[width=0.66\linewidth]{images/mrautomatr_doc_page} \end{center}

\hypertarget{set-up-the-parameters}{%
\chapter{Set up the parameters}\label{set-up-the-parameters}}

\hypertarget{organize-your-model-outputs}{%
\section{Organize your model outputs}\label{organize-your-model-outputs}}

Before you run any R code, you need to make sure that the parameters for the report are correctly specified.

First, copy and paste all \textbf{currently available} final Mplus models (only the \texttt{.out} files) into one folder (e.g., a folder called \texttt{Measurement\ report/Models} somewhere on Box). This includes:

\begin{itemize}
\tightlist
\item
  EFA models
\item
  CFA models
\item
  Treatment invariance models
\item
  Age invariance models
\item
  Gender invariance models
\item
  Longitudinal invariance models
\end{itemize}

\hypertarget{a-very-important-note-on-naming-rules}{%
\section{A very important note on naming rules!!!}\label{a-very-important-note-on-naming-rules}}

You need to name your file, items, and constructs properly in order for the wave tags shown properly in the document.

For file names, always have a wave tag right after your measure. For example, \texttt{CSRL1\_cfa\_1c\_fsamp.out}, \texttt{RSQ2\_CFA2-ie19.out}. If you only have one wave, use \texttt{1} as your wave tag.

For items and constructs (in your data, in Mplus files, and in the Excel sheet), \textbf{always add an underscore and a wave tag (\_\#)} in the subscale section to both the construct name and the item names in your excel sheet (e.g., \texttt{AANXDEP\_1}, \texttt{PSRA\_1}, \texttt{HAB\_AB\_2}).

These wave tags get picked up by \texttt{mrautomatr} to index the waves in plots and tables in your document.

\hypertarget{fill-in-the-excel-template}{%
\section{Fill in the Excel template}\label{fill-in-the-excel-template}}

Second, fill in the Excel sheet template that we provide. You can find this template (\texttt{input\_template.xlsx}) located in \texttt{inst/templates/} in the \href{https://github.com/nyuglobalties/mrautomatr/blob/master/inst/templates/input_template.xlsx}{package GitHub repository}. Simply hit \texttt{Download} to download the template and store somewhere in your computer.

Alternatively, this Excel file is downloaded along with the \texttt{mrautomatr} package. You can also find its file path on your computer by running \texttt{system.file("templates",\ "input\_template.xlsx",\ package\ =\ "mrautomatr")} in R. Copy the file path. If you are a Mac user, go to your Finder and hit Go - Go to Folder in your toolbar, and paste the path there and hit Go. Then copy and paste the excel file somewhere in your computer (e.g., a folder called \texttt{Measurement\ report/template}).

\begin{center}\includegraphics[width=0.66\linewidth]{images/go_to_folder1} \end{center}

\begin{center}\includegraphics[width=0.66\linewidth]{images/go_to_folder2} \end{center}

In this template, you need to manually type in the following parameters. For any of the files that are not available temporarily, you can leave blank and still be able to generate the report (with error messages shown in the Word document telling you that you need to specify more parameters/fix certain things to have a full report).

\hypertarget{tab-1-path}{%
\subsection{\texorpdfstring{Tab 1: \texttt{path}}{Tab 1: path}}\label{tab-1-path}}

A shorthand to get file path on Mac: go to the path/file and hit \texttt{command\ +\ option\ +\ C}.

If you are using Box, we recommend typing in an R function from \texttt{mrautomatr} -- \texttt{box\_path("...")} in the \texttt{path} tab in the excel sheet. The \texttt{...} part is whatever sub-folders on your computer under the general Box folder. Examples are \texttt{box\_path("Box\ 3EA\ Team\ Folder/Data\ Management")}, or \texttt{box\_path("Peru/Data/Full")}. This is because different people can have different access to Box folders at different time (e.g.~you only had access to \texttt{Data\ Management} in March, but gained access to \texttt{Box\ 3EA\ Team\ Folder} in April), and if you only specify the local path on your computer, you will need to re-specify it in the excel sheet whenever your level of access changes. \texttt{box\_path(...)} prevents this from happening.

\begin{itemize}
\item
  \texttt{year} indicates the study year.
\item
  \texttt{measure} indicates the measure name.
\item
  \texttt{data\_file\_path} should be wherever the final master data is located. It will be used to calculate summary statistics, bivariate correlations, and reliability statistics. Our tool currently takes the following data formats: \texttt{.csv}, \texttt{.xlsx}, \texttt{.dta}.
\item
  \texttt{fs\_data\_file\_path} refers to the file path where the tabular data of the Mplus-generated factor scores is saved. Because Mplus \textbf{does not} generate a spreadsheet for you, you will need to:

  \begin{itemize}
  \item
    \textbf{(1) copy and paste the factor scores into an Excel sheet, and}
  \item
    \textbf{(2) insert the first row and name the variables exactly the same as they are in your master dataset and in your other Mplus models. Remember to add the \_\# wave tags.}
  \item
    \textbf{(3) save the sheet either as a .csv or an .xlsx file.}
  \end{itemize}
\item
  \texttt{model\_file\_path} leads you to the folder where all the Mplus outputs are located.
\end{itemize}

\hypertarget{tab-2-subscale}{%
\subsection{\texorpdfstring{Tab 2: \texttt{subscale}}{Tab 2: subscale}}\label{tab-2-subscale}}

\begin{itemize}
\item
  The first row should contain the subscale/factor names. They should be the same as the ones in your Mplus models.
\item
  For each subscale/factor, list the items. The rows can be of unequal length (i.e.~you can leave blanks for subscales with smaller number of items).
\item
  These are specified to generate reliability estimates from the master dataset.
\item
  \textbf{Always} add an underscore and a wave tag (\texttt{\_\#}) in the \texttt{subscale} section to both the construct name and the item names in your excel sheet (e.g., \texttt{AANXDEP\_1}). Whatever numbers comes after ``\_'' will be shown in the table as the wave tag. And the wave tag also needs to be added for the variables in the master dataset. Not sure if this is easy to do in Mplus, but you can certainly export another master dataset after running things like \texttt{names(dat){[}1:10{]}\ \textless{}-\ paste(names(dat){[}1:10{]},\ "\_1",\ sep\ =\ "")} on the variables you want to modify.
\end{itemize}

The Omega reliability coefficients will likely not show up if the wave tags are misspecified. If you have already run quite many models and generated multiple reports, I'd suggest using the \texttt{get.omega.bywave()} in the\texttt{mrautomatr} and manually fill in the omegas in your document. E.g.:

\begin{verbatim}
get.omega.bywave(model = "xxx.out", 
                 path = "/Users/xxx")
\end{verbatim}

\hypertarget{tab-3-model}{%
\subsection{\texorpdfstring{Tab 3: \texttt{model}}{Tab 3: model}}\label{tab-3-model}}

\begin{itemize}
\item
  This specifies all necessary Mplus model names (i.e.~\texttt{xxx.out}).
\item
  List all available models in the order of waves (e.g.~wave 1 before wave 2).
\item
  There is no restrictions on the file names, but please follow the \href{https://nyu.box.com/s/ate5l7wmw164u7xjg3g8x1vrfhwnt0ax}{naming rules} for reproducibility purposes.
\end{itemize}

\hypertarget{tab-4-description}{%
\subsection{\texorpdfstring{Tab 4: \texttt{description}}{Tab 4: description}}\label{tab-4-description}}

\begin{itemize}
\item
  This is specified to have a description of the items at the beginning of the report.
\item
  You can format this tab in any ways that you like, but the caveat is that (1) the first row will be taken as the header and set to bold, and (2) you cannot merge cells.
\end{itemize}

\hypertarget{summary}{%
\section{Summary}\label{summary}}

\begin{longtable}[]{@{}ll@{}}
\toprule
\begin{minipage}[b]{(\columnwidth - 1\tabcolsep) * \real{0.25}}\raggedright
Variable name\strut
\end{minipage} & \begin{minipage}[b]{(\columnwidth - 1\tabcolsep) * \real{0.75}}\raggedright
Description\strut
\end{minipage}\tabularnewline
\midrule
\endhead
\begin{minipage}[t]{(\columnwidth - 1\tabcolsep) * \real{0.25}}\raggedright
\texttt{year}\strut
\end{minipage} & \begin{minipage}[t]{(\columnwidth - 1\tabcolsep) * \real{0.75}}\raggedright
Study site and year\strut
\end{minipage}\tabularnewline
\begin{minipage}[t]{(\columnwidth - 1\tabcolsep) * \real{0.25}}\raggedright
\texttt{measure}\strut
\end{minipage} & \begin{minipage}[t]{(\columnwidth - 1\tabcolsep) * \real{0.75}}\raggedright
Measure name\strut
\end{minipage}\tabularnewline
\begin{minipage}[t]{(\columnwidth - 1\tabcolsep) * \real{0.25}}\raggedright
\texttt{data\_file\_path}\strut
\end{minipage} & \begin{minipage}[t]{(\columnwidth - 1\tabcolsep) * \real{0.75}}\raggedright
Local file path to the master dataset on your own computer\strut
\end{minipage}\tabularnewline
\begin{minipage}[t]{(\columnwidth - 1\tabcolsep) * \real{0.25}}\raggedright
\texttt{fs\_data\_file\_path}\strut
\end{minipage} & \begin{minipage}[t]{(\columnwidth - 1\tabcolsep) * \real{0.75}}\raggedright
Local file path to the factor score dataset on your own computer\strut
\end{minipage}\tabularnewline
\begin{minipage}[t]{(\columnwidth - 1\tabcolsep) * \real{0.25}}\raggedright
\texttt{model\_file\_path}\strut
\end{minipage} & \begin{minipage}[t]{(\columnwidth - 1\tabcolsep) * \real{0.75}}\raggedright
Local file path to all the Mplus .out files\strut
\end{minipage}\tabularnewline
\begin{minipage}[t]{(\columnwidth - 1\tabcolsep) * \real{0.25}}\raggedright
\texttt{subscale}\strut
\end{minipage} & \begin{minipage}[t]{(\columnwidth - 1\tabcolsep) * \real{0.75}}\raggedright
Subscales and their corresponding items\strut
\end{minipage}\tabularnewline
\begin{minipage}[t]{(\columnwidth - 1\tabcolsep) * \real{0.25}}\raggedright
\texttt{model\_efa}\strut
\end{minipage} & \begin{minipage}[t]{(\columnwidth - 1\tabcolsep) * \real{0.75}}\raggedright
EFA models\strut
\end{minipage}\tabularnewline
\begin{minipage}[t]{(\columnwidth - 1\tabcolsep) * \real{0.25}}\raggedright
\texttt{model\_cfa}\strut
\end{minipage} & \begin{minipage}[t]{(\columnwidth - 1\tabcolsep) * \real{0.75}}\raggedright
CFA models\strut
\end{minipage}\tabularnewline
\begin{minipage}[t]{(\columnwidth - 1\tabcolsep) * \real{0.25}}\raggedright
\texttt{model\_inv\_tx}\strut
\end{minipage} & \begin{minipage}[t]{(\columnwidth - 1\tabcolsep) * \real{0.75}}\raggedright
Treatment invariance models\strut
\end{minipage}\tabularnewline
\begin{minipage}[t]{(\columnwidth - 1\tabcolsep) * \real{0.25}}\raggedright
\texttt{model\_inv\_gender}\strut
\end{minipage} & \begin{minipage}[t]{(\columnwidth - 1\tabcolsep) * \real{0.75}}\raggedright
Gender invariance models\strut
\end{minipage}\tabularnewline
\begin{minipage}[t]{(\columnwidth - 1\tabcolsep) * \real{0.25}}\raggedright
\texttt{model\_inv\_age}\strut
\end{minipage} & \begin{minipage}[t]{(\columnwidth - 1\tabcolsep) * \real{0.75}}\raggedright
Age invariance models\strut
\end{minipage}\tabularnewline
\begin{minipage}[t]{(\columnwidth - 1\tabcolsep) * \real{0.25}}\raggedright
\texttt{model\_inv\_lg}\strut
\end{minipage} & \begin{minipage}[t]{(\columnwidth - 1\tabcolsep) * \real{0.75}}\raggedright
Longitudinal invariance model\strut
\end{minipage}\tabularnewline
\begin{minipage}[t]{(\columnwidth - 1\tabcolsep) * \real{0.25}}\raggedright
\texttt{description}\strut
\end{minipage} & \begin{minipage}[t]{(\columnwidth - 1\tabcolsep) * \real{0.75}}\raggedright
Detailed item descriptions\strut
\end{minipage}\tabularnewline
\bottomrule
\end{longtable}

\hypertarget{generate-the-report}{%
\chapter{Generate the report}\label{generate-the-report}}

After carefully setting your parameters, you can now generate your report!

There are three ways to generate/knit reports:

\begin{enumerate}
\def\labelenumi{\arabic{enumi}.}
\item
  Generate one report for one measure using the default settings \texttt{render\_report()}
\item
  Generate one report for one measure using customized settings by the users \texttt{render\_report\_manual()}
\item
  Generate multiple separate reports for multiple measures using default settings \texttt{render\_report\_multiple()}
\end{enumerate}

After generating the report, make sure to rename it and manually edit the sections that are text-heavy. The renaming is \textbf{necessary} because you may accidentally overwrite your manual edits if you regenerate the report in R.

Rmarkdown is not powerful yet to allow back-translation from Word to R codes, so your manual changes in Word will \textbf{NOT} be reflected in the R codes when you regenerate the report for some reasons (e.g.~wrong file names). Therefore, we recommend finalizing the tables and plots before you write texts in the Word document (or you can just store the texts in another and move them over to the master report whenever you feel ready). If you've already written extensively in a knitted report and want to fix certain small sections/add some numbers, you can run the individual functions in R and manually make the changes.

\textbf{Note.} Error messages are shown in the knitted document in their corresponding section. They are usually about certain parameters not being specified. Other error messages should also be pretty understandable.

\textbf{Note.} Warning messages are not shown in the knitted document. They are usually okay to ignore as they often come from the \texttt{MplusAutomation} package failing to read certain bits of the Mplus output that are not quite important to generating the reports. They may also come from that WRMR is reported instead of SRMR due to Mplus version conflict. If it's telling you an error related to \texttt{xxx\_fscores.csv}, simply ignore it (see this \href{https://github.com/nyuglobalties/mrautomatr/issues/20}{GitHub issue} for an explanation). You can also set \texttt{printwarning\ =\ TRUE} in rendering the reports (see the functions below) to have warnings printed in the documents too, or run \texttt{warnings()} in R to get all warning messages.

\hypertarget{render_report}{%
\section{\texorpdfstring{\texttt{render\_report()}}{render\_report()}}\label{render_report}}

This function renders one report for the specified measure.

Run \texttt{?render\_report()} to see what each argument represents. \texttt{parameters} allows you to specify a list of parameters to control the \texttt{params} section in the Rmarkdown template, you can omit this argument to use the default settings. See the bottom of this page for explanations on these parameters.

Example:

\begin{verbatim}
render_report(output_dir = "/Users/michaelfive/Google Drive/NYU/3EA/test",
              output_file = "Report_lebanon_cs.docx",
              parameters = list(
                       # set R code print options
                       printcode = FALSE,
                       printwarning = FALSE,
                       storecache = FALSE,

                       # set report overall parameters
                       template = "/Users/michaelfive/Google Drive/NYU/3EA/test/input_template_lebanon_cs.xlsx",
                       set_title = "Lebanon Year 1 (2016-2017)",
                       set_author = "Jane Doe",

                       # select report sections
                       item = TRUE,
                       descriptive = TRUE,
                       ds_plot = TRUE,
                       correlation_matrix_lg = TRUE,
                       correlation_matrix_bivar = TRUE,
                       correlation_matrix_item = FALSE, # BE CAREFUL! This might crash the document.
                       efa_screeplot = TRUE,
                       cfa_model_fit = TRUE,
                       cfa_model_plot = TRUE,
                       cfa_model_parameters = TRUE,
                       cfa_model_thresholds = FALSE, # default is to mute the thresholds as the table can get very long
                       cfa_r2 = TRUE,
                       internal_reliability = TRUE,
                       summary_item_statistics = TRUE,
                       item_total_statistics = TRUE,
                       inv_tx = TRUE,
                       inv_gender = TRUE,
                       inv_age = TRUE,
                       inv_lg = TRUE
                       ))
\end{verbatim}

\hypertarget{render_report_manual}{%
\section{\texorpdfstring{\texttt{render\_report\_manual()}}{render\_report\_manual()}}\label{render_report_manual}}

Run \texttt{?render\_report\_manual()} to see what each argument represents.

Example:

\begin{verbatim}
render_report_manual(output_file = "Report_lebanon_cs.docx",
                     output_dir = "/Users/michaelfive/Google Drive/NYU/3EA/test")
\end{verbatim}

This function opens a Shiny web page where you can click/unclick sections you'd like to include/exclude in the report (see descriptions below). It also renders one report for the specified measure.

Click Save to start rendering the report, click Cancel to stop the web app and go back to R and hit the Esc button to exit the Shiny web session.

\begin{center}\includegraphics[width=0.66\linewidth]{images/render_report_manual1} \end{center}

\begin{center}\includegraphics[width=0.66\linewidth]{images/render_report_manual2} \end{center}

\hypertarget{render_report_multiple}{%
\section{\texorpdfstring{\texttt{render\_report\_multiple()}}{render\_report\_multiple()}}\label{render_report_multiple}}

Run \texttt{?render\_report\_multiple()} to see what each argument represents. This function renders multiple reports at the same time with parameters globally set for all reports.

Example:

\begin{verbatim}
render_report_multiple(input_dir = "/Users/michaelfive/Google Drive/NYU/3EA/test",
                       templates = c("input_template_lebanon_cs.xlsx",
                                     "input_template_niger_psra.xlsx"),
                       output_dir = "/Users/michaelfive/Google Drive/NYU/3EA/test",
                       # set parameters globally for all documents
                       parameters = list(
                       # set R code print options
                       printcode = FALSE,
                       printwarning = FALSE,
                       storecache = FALSE,

                       # set report overall parameters
                       set_author = "Jane Doe",
                       # report title comes from the `year` tab in each excel template

                       # select report sections
                       item = TRUE,
                       descriptive = TRUE,
                       ds_plot = TRUE,
                       correlation_matrix_lg = TRUE,
                       correlation_matrix_bivar = TRUE,
                       correlation_matrix_item = FALSE, # BE CAREFUL! This might crash the document.
                       efa_screeplot = TRUE,
                       cfa_model_fit = TRUE,
                       cfa_model_plot = TRUE,
                       cfa_model_parameters = TRUE,
                       cfa_model_thresholds = FALSE, # default is to mute the thresholds as the table can get very long
                       cfa_r2 = TRUE,
                       internal_reliability = TRUE,
                       summary_item_statistics = TRUE,
                       item_total_statistics = TRUE,
                       inv_tx = TRUE,
                       inv_gender = TRUE,
                       inv_age = TRUE,
                       inv_lg = TRUE
                       ))
\end{verbatim}

\begin{longtable}[]{@{}ll@{}}
\toprule
\begin{minipage}[b]{(\columnwidth - 1\tabcolsep) * \real{0.13}}\raggedright
Parameters\strut
\end{minipage} & \begin{minipage}[b]{(\columnwidth - 1\tabcolsep) * \real{0.87}}\raggedright
Description\strut
\end{minipage}\tabularnewline
\midrule
\endhead
\begin{minipage}[t]{(\columnwidth - 1\tabcolsep) * \real{0.13}}\raggedright
\texttt{printcode}\strut
\end{minipage} & \begin{minipage}[t]{(\columnwidth - 1\tabcolsep) * \real{0.87}}\raggedright
whether you'd like R codes to be printed in your document\strut
\end{minipage}\tabularnewline
\begin{minipage}[t]{(\columnwidth - 1\tabcolsep) * \real{0.13}}\raggedright
\texttt{printwarning}\strut
\end{minipage} & \begin{minipage}[t]{(\columnwidth - 1\tabcolsep) * \real{0.87}}\raggedright
whether you'd like to print warnings in running the codes\strut
\end{minipage}\tabularnewline
\begin{minipage}[t]{(\columnwidth - 1\tabcolsep) * \real{0.13}}\raggedright
\texttt{storecache}\strut
\end{minipage} & \begin{minipage}[t]{(\columnwidth - 1\tabcolsep) * \real{0.87}}\raggedright
whether you'd like to store \texttt{knitr} cache (only for programming purposes, see \href{https://bookdown.org/yihui/rmarkdown-cookbook/cache.html}{here})\strut
\end{minipage}\tabularnewline
\begin{minipage}[t]{(\columnwidth - 1\tabcolsep) * \real{0.13}}\raggedright
\texttt{set\_title}\strut
\end{minipage} & \begin{minipage}[t]{(\columnwidth - 1\tabcolsep) * \real{0.87}}\raggedright
title\strut
\end{minipage}\tabularnewline
\begin{minipage}[t]{(\columnwidth - 1\tabcolsep) * \real{0.13}}\raggedright
\texttt{set\_author}\strut
\end{minipage} & \begin{minipage}[t]{(\columnwidth - 1\tabcolsep) * \real{0.87}}\raggedright
author\strut
\end{minipage}\tabularnewline
\begin{minipage}[t]{(\columnwidth - 1\tabcolsep) * \real{0.13}}\raggedright
\texttt{template}\strut
\end{minipage} & \begin{minipage}[t]{(\columnwidth - 1\tabcolsep) * \real{0.87}}\raggedright
parameter template file path\strut
\end{minipage}\tabularnewline
\begin{minipage}[t]{(\columnwidth - 1\tabcolsep) * \real{0.13}}\raggedright
\texttt{item}\strut
\end{minipage} & \begin{minipage}[t]{(\columnwidth - 1\tabcolsep) * \real{0.87}}\raggedright
print item descriptions\strut
\end{minipage}\tabularnewline
\begin{minipage}[t]{(\columnwidth - 1\tabcolsep) * \real{0.13}}\raggedright
\texttt{descriptive}\strut
\end{minipage} & \begin{minipage}[t]{(\columnwidth - 1\tabcolsep) * \real{0.87}}\raggedright
print descriptive statistics table\strut
\end{minipage}\tabularnewline
\begin{minipage}[t]{(\columnwidth - 1\tabcolsep) * \real{0.13}}\raggedright
\texttt{ds\_plot}\strut
\end{minipage} & \begin{minipage}[t]{(\columnwidth - 1\tabcolsep) * \real{0.87}}\raggedright
print descriptive statistics histograms\strut
\end{minipage}\tabularnewline
\begin{minipage}[t]{(\columnwidth - 1\tabcolsep) * \real{0.13}}\raggedright
\texttt{correlation\_matrix\_lg}\strut
\end{minipage} & \begin{minipage}[t]{(\columnwidth - 1\tabcolsep) * \real{0.87}}\raggedright
print factor-level correlation matrix from longitudinal invariance model\strut
\end{minipage}\tabularnewline
\begin{minipage}[t]{(\columnwidth - 1\tabcolsep) * \real{0.13}}\raggedright
\texttt{correlation\_matrix\_bivar}\strut
\end{minipage} & \begin{minipage}[t]{(\columnwidth - 1\tabcolsep) * \real{0.87}}\raggedright
print factor-level correlation matrix from master dataset\strut
\end{minipage}\tabularnewline
\begin{minipage}[t]{(\columnwidth - 1\tabcolsep) * \real{0.13}}\raggedright
\texttt{correlation\_matrix\_item}\strut
\end{minipage} & \begin{minipage}[t]{(\columnwidth - 1\tabcolsep) * \real{0.87}}\raggedright
print item-level correlation matrix from master dataset (set to \texttt{FALSE} because correlations among dozens of items may be unnecessary)\strut
\end{minipage}\tabularnewline
\begin{minipage}[t]{(\columnwidth - 1\tabcolsep) * \real{0.13}}\raggedright
\texttt{efa\_screeplot}\strut
\end{minipage} & \begin{minipage}[t]{(\columnwidth - 1\tabcolsep) * \real{0.87}}\raggedright
print EFA screeplot at all waves\strut
\end{minipage}\tabularnewline
\begin{minipage}[t]{(\columnwidth - 1\tabcolsep) * \real{0.13}}\raggedright
\texttt{cfa\_model\_fit}\strut
\end{minipage} & \begin{minipage}[t]{(\columnwidth - 1\tabcolsep) * \real{0.87}}\raggedright
print CFA model fits at all waves\strut
\end{minipage}\tabularnewline
\begin{minipage}[t]{(\columnwidth - 1\tabcolsep) * \real{0.13}}\raggedright
\texttt{cfa\_model\_plot}\strut
\end{minipage} & \begin{minipage}[t]{(\columnwidth - 1\tabcolsep) * \real{0.87}}\raggedright
print CFA model path diagram (for the first specified CFA model; i.e.~Time 1; assuming factor structure does not change)\strut
\end{minipage}\tabularnewline
\begin{minipage}[t]{(\columnwidth - 1\tabcolsep) * \real{0.13}}\raggedright
\texttt{cfa\_model\_parameters}\strut
\end{minipage} & \begin{minipage}[t]{(\columnwidth - 1\tabcolsep) * \real{0.87}}\raggedright
print CFA model parameters at all waves (factor loadings)\strut
\end{minipage}\tabularnewline
\begin{minipage}[t]{(\columnwidth - 1\tabcolsep) * \real{0.13}}\raggedright
\texttt{cfa\_model\_thresholds}\strut
\end{minipage} & \begin{minipage}[t]{(\columnwidth - 1\tabcolsep) * \real{0.87}}\raggedright
print the thresholds in CFA model parameters at all waves; default is \texttt{FALSE} in order to mute the thresholds as the table can get very long\strut
\end{minipage}\tabularnewline
\begin{minipage}[t]{(\columnwidth - 1\tabcolsep) * \real{0.13}}\raggedright
\texttt{cfa\_r2}\strut
\end{minipage} & \begin{minipage}[t]{(\columnwidth - 1\tabcolsep) * \real{0.87}}\raggedright
print CFA model R-squared at all wave\strut
\end{minipage}\tabularnewline
\begin{minipage}[t]{(\columnwidth - 1\tabcolsep) * \real{0.13}}\raggedright
\texttt{internal\_reliability}\strut
\end{minipage} & \begin{minipage}[t]{(\columnwidth - 1\tabcolsep) * \real{0.87}}\raggedright
print estimates of internal reliability (Cronbach's Alpha and McDonald's Omega, descriptions of the other indices can be found \href{https://personality-project.org/r/html/alpha.html}{here})\strut
\end{minipage}\tabularnewline
\begin{minipage}[t]{(\columnwidth - 1\tabcolsep) * \real{0.13}}\raggedright
\texttt{summary\_item\_statistics}\strut
\end{minipage} & \begin{minipage}[t]{(\columnwidth - 1\tabcolsep) * \real{0.87}}\raggedright
print summary item statistics (descriptions of the other indices can be found \href{https://personality-project.org/r/html/alpha.html}{here})\strut
\end{minipage}\tabularnewline
\begin{minipage}[t]{(\columnwidth - 1\tabcolsep) * \real{0.13}}\raggedright
\texttt{item\_total\_statistics}\strut
\end{minipage} & \begin{minipage}[t]{(\columnwidth - 1\tabcolsep) * \real{0.87}}\raggedright
print total item statistics (descriptions of the other indices can be found \href{https://personality-project.org/r/html/alpha.html}{here})\strut
\end{minipage}\tabularnewline
\begin{minipage}[t]{(\columnwidth - 1\tabcolsep) * \real{0.13}}\raggedright
\texttt{inv\_tx}\strut
\end{minipage} & \begin{minipage}[t]{(\columnwidth - 1\tabcolsep) * \real{0.87}}\raggedright
print model fits for treatment invariance models at all waves\strut
\end{minipage}\tabularnewline
\begin{minipage}[t]{(\columnwidth - 1\tabcolsep) * \real{0.13}}\raggedright
\texttt{inv\_gender}\strut
\end{minipage} & \begin{minipage}[t]{(\columnwidth - 1\tabcolsep) * \real{0.87}}\raggedright
print model fits for gender invariance models at all waves\strut
\end{minipage}\tabularnewline
\begin{minipage}[t]{(\columnwidth - 1\tabcolsep) * \real{0.13}}\raggedright
\texttt{inv\_age}\strut
\end{minipage} & \begin{minipage}[t]{(\columnwidth - 1\tabcolsep) * \real{0.87}}\raggedright
print model fits for age invariance models at all waves\strut
\end{minipage}\tabularnewline
\begin{minipage}[t]{(\columnwidth - 1\tabcolsep) * \real{0.13}}\raggedright
\texttt{inv\_lg}\strut
\end{minipage} & \begin{minipage}[t]{(\columnwidth - 1\tabcolsep) * \real{0.87}}\raggedright
print model fit for the longitudinal invariance model\strut
\end{minipage}\tabularnewline
\bottomrule
\end{longtable}

\hypertarget{individual-functions}{%
\chapter{Individual functions}\label{individual-functions}}

If you are an R user who wishes to run individual functions in this package to get results in R instead of Word, you can check the help pages of those functions by running \texttt{?mrautomatr}.

  \bibliography{book.bib,packages.bib}

\end{document}
